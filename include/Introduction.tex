% CREATED BY MAGNUS GUSTAVER, 2020
\chapter{Introduction}
%"AMR is bad and sad"

% Elaborate on this a bit more. Resistance is not defined yet. What resistances can we talk about? How does it relate to the environment?
% Common treatments such as ... for ....
% Antibiotics released into aquatic ecosystems contribute to 

Antibiotic resistance is one of the top threats to human health according to the WHO \cite{worldhealthorganization2023AntimicrobialResistance}, and bacterial antimicrobial resistance (AMR) is estimated to have caused 1.14 million deaths in 2021 \cite{naghavi2024GlobalBurdenBacterial}.
The term antimicrobial resistance describes when a bacterial species is less susceptible to antimicrobial treatments, and as a result increases the difficulty of treating the infection, which can lead to an increase in disease spread, illness and potential risk of death \cite{worldhealthorganization2023AntimicrobialResistance}.
%Infections with antibiotic resistant bacteria are less likely to be
% Microorganims might become resistant to certain antibiotics, reducing the efficacy of the treatments, by exposure to 
% The cause of these deaths are  
It is estimated that by the year 2050, resistant infections could be associated with more than 8 million deaths anually \cite{naghavi2024GlobalBurdenBacterial}.
Humans contribute to the spread of antibiotic resistance by improper use in both clinical and agricultural settings \cite{marston2016AntimicrobialResistance, ding2023SpreadAntibioticResistance}. 
In clinical settings, this includes cases where antibiotics are improperly chosen for the infection at hand, or assigned at an inappropriate dosage.
In agriculture, antibiotics are used as promotors of growth, rather than as a treatment for infection.
%These uses also contribute to the spread of antimicrobial resistance, where the resistant microbes as well as the antibiotics reach the environment via excretions,
These uses also contribute to antibiotics in the enviroment, where it is excreted from humans and animals after use.
Antibiotics in the environment are suggested to play a role in increased antibiotic resistance, and the environment can act as a transmission agent which spreads resistant bacteria to new hosts \cite{larsson2022AntibioticResistanceEnvironment}.
Microplastics are present in soil, marine and freshwater ecosystems, and has been recognized as an important source of pollution \cite{ziani2023MicroplasticsRealGlobal}. 
The plastisphere, the microbial community which colonize the surface of microplastics, has been shown to enrich potentially pathogenic bacteria that carry antibiotic resistance genes (ARGs) \cite{liu2021MicroplasticsAreHotspot, wu2019SelectiveEnrichmentBacterial}.
% the emergence of antibiotic resistance is not fully understood \cite{kummerer2009AntibioticsAquaticEnvironment BRO 2009?
This highlights the need for a deeper understanding of the mechanisms by which AMR is connected to microplastics and the environment, in order for the threat of antibiotic resistant infections to be mitigated. 

%TODO: why point mutations, not well studied?

% AMR bad
% - Why?
% - How?
% Plastics increase AMR
% - How?
% - Why more plastics?
% Plastisphere - in theory instead?

%Base it on the project plan

%Social/ethical considerations?? Only in planning report? Should be motivated in the planning report?

%\section{Background} \todo{Skip this? Instead use theory as background?}

\section{Aim} %Renamed from purpose
The aim of the project was to assess the presence of ARGs in the plastisphere and identify specific point mutations which can lead to decreased susceptibility of bacteria to antibiotics.
To do so, bioinformatic methods was used to screen publically available metagenomic data for antibiotic resistance genes in microorganims which colonize microplastics. This will then be compared with those which inhibit the surrounding water as well as other natural and non-natural substrates.

Our hypothesis was thus that microplastics would increase the mutation frequency of specific antibiotic resistance mutations.

The findings from the current project enhances the current knowledge regarding AMR and how it is connected to microplastics in aquatic ecosystems.
%as well as improve the ability for us to monitor the presence of AMR in the future. 

% - Public metagenomic data
% - Examine the presence of ARGs in the plastisphere, which can lead to resistance
% - Screen microorganims which colonize microplastics, and compare to surrounding water and non-plastic substrates: wood/glass
% - Estimate point mutations of plastisphere bacteria, compare to ...
% Use MuMaMe, specifically developed to detect such genetic changes

% Findings could aid in enhancing the current knowledge regarding AMR, as well as enable us to monitor the presence of AMR in the future.

%\section{Goals} % Include in Aim?

% \section{Limitations}
% - Only used CARD as databse, only those present in it
% The ARGs analyzed are a part of version 4.0.0 of the Comprehensive Antibiotic Resistance Database (CARD)\cite{alcock2023CARD2023Expanded}, released in December 2024. This limits the results to only include genes present in that database, however it is the preferable choice of database for the study of resistance genes and mutations\cite{papp2022ReviewComparisonAntimicrobial}.
% The tool MuMaMe only looks at amino acid substitutions (missense mutations), and not insertions/deletions (frameshift mutations) or terminations (nonsense mutations)\cite{liu2024Chapter106Virus}. 
%The tool MuMaMe only searches and matches reads to point mutations which , and not deletions, insertions or terminations. 
% Refer to something that it is most important for }

%which were screened for was accesesd from the Comprehensive Antibiotic Resistance Database (CARD)\cite{alcock2023CARD2023Expanded}.
%The database for the point mutations which were to be screened for in the samples  

%Limitations of the project include the amount of data, which was 
