% Created by Kyriaki Antoniadou-Plytaria, 2021
\thispagestyle{plain}			% Supress header
% \vspace*{3.5cm}
% \section*{List of Acronyms}
\chapter*{List of Acronyms}
Below is the list of acronyms that have been used throughout this thesis listed in alphabetical order:
\vspace*{1.0cm}

%\begin{table}[H]
%\centering
\begin{tabular}{p{3cm}p{12cm}}
AMR & Anti-microbial resistance \\
ARG & Antibiotic resistance genes \\
HDPE & High-density polyethylene \\
LDPE & Low-density polyethylene \\
HGT & Horizontal gene transfer \\
PBAT & Polybutylene adipate terephthalate (Biodegradable) \\
PF & PE-fiber \\
PE & Polyethylene \\
PFP & PE-fiber-PE \\
PHA & polyhydroxyalkanoate (Biodegradable) \\
PHB & Polyhydroxybutyrate (Biodegradable) \\
PHBV & poly(3-hydroxybutyrate-co-3-hydroxyvalerate) (Biodegradable) \\
PLA & Polylactic acid (Biodegradable) \\
PVC & Polyvinyl chloride \\ 
SNP & Single Nucleotide Polymorphism
\end{tabular}
%\end{table}

% Comments Máté
% - Start result with summary of the data processed, how many resistance genes were mapped to card et.c. Then describe figure 2.1 which is unclear
% - Don't forget your opponent is not familiar with this topic, so "serve" your results as you would in a documentary, for a broader audience.
% - If you use the word significance, you need to add e.g. (Wilcoxon test: p < 0.05
% - Only one caption is enough for combined plots, there you can mention (a) text, and (b) text.
% - You need to be extremely clear about the differences between the mean percent mutation and the number of hits section.
% - Taxonomic names names: capitalized first letter and italics
% - mention that you collected data from various ecosysems which all had different substrates

% Máté
% - remove leaf/rock etc from all analyses
% x point mutation on either forward/reverse -> 6 dots intsead of 3
% x check coordinates, specify east/west
% x working code (notebook) as pdf in appendix



\newpage				% Create empty back of side
\thispagestyle{empty}
\mbox{}

