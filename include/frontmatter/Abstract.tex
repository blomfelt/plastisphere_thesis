% CREATED BY MAGNUS GUSTAVER, 2020
Antibiotic Resistance Mutations in Aquatic Plastisphere Microbiomes\\
\\
FELIX BLOMFELT\\
Department of Life Sciences\\
Chalmers University of Technology\\
\if\InstitutionLocation G
University of Gothenburg\\
\fi
\setlength{\parskip}{0.5cm}

\thispagestyle{plain}			% Supress header 
\setlength{\parskip}{0pt plus 1.0pt}
\section*{Abstract}

% \begin{itemize}
    % \item ll - start/stop compiling the document, turns on auto compilation.
    % \item lk - stop compilation process. 
    % \item lc - clear auxiliary files 
    % \item lv - forward search, will open compiled pdf. Some pdf viewers support the ability to jump to the current location in the pdf. 
    % \item le - open/close quickfix window. Pressing enter on the error moves the cursor to the line containing the error. 
    % % TO-DO example of to-do-note, see also: 
    % \item todo{To create a visible todo-note}
    % \item lt - open table of contents for the file. Includes section, labels, references, and TODOs. 
    % \item [[, [], ][, ]] - move between section boundaries. 
    % \item { - move to previous empty line, } to the next. 
    % \item \% - move to next/matching bracket. 
    % \item ic - excludes the backslash of latex commands. 
    % \item ac - includes the backslash of latex commands. 
    % \item cse - change surrounding environment, e.g. align to equation. 
    % \item tsd - toggles between () and \( \left( \right) \) in enviroments.
    % \item tse - toggles the \* in enviroments.
    % \item c - change 
    % \item d - delete 
    % \item :help vimtex - open the help page
    % \item S - delete whole line and begin insert
    % \item s - delete character and begin insert
% 
% \end{itemize}
% 
% 
% \section{TODO}
% \begin{itemize}
    % \item Boxplots
        % \begin{itemize}
            % \item Samples
            % \item Mutations
            % \item Sampletype/substrate
            % \item Study and Coords - mention but no results?
            % \item Ecosystem - mention but no results?
        % \end{itemize}
    % \item[x] Wilcoxon test
        % \begin{itemize}
            % \item[x] Pseudo-mean
            % \item[x] Paired or not
        % \end{itemize}
    % \item[x] Random Forest
            % %\item "Permuting a useful variable, tend to give relatively large decrease in mean gini-gain. GINI importance is closely related to the local decision function, that random forest uses to select the best available split. Therefore, it does not take much extra time to compute. On the other hand, mean gini-gain in local splits, is not necessarily what is most useful to measure, in contrary to change of overall model performance. Gini importance is overall inferior to (permutation based) variable importance as it is relatively more biased, more unstable and tend to answer a more indirect question" - see https://stats.stackexchange.com/questions/197827/how-to-interpret-mean-decrease-in-accuracy-and-mean-decrease-gini-in-random-fore
    % \item Weird point-plot, see fig \ref{pointplot_mutations}
        % \subitem Use it or not? Kinda relevant
    % \item Taxonomic Composition - mention above after metaxaQR?
% \end{itemize}
% 
Antibiotic resistance, where pathogens are less susceptible to treatments with antibiotics, is one of the top threats to human health, and antibiotic resistance is estimated to have caused more than one million deaths in 2021. 
% Too large jump.
% Potentially start by talking about the pathogens and then mention that certain pathogens can be enriched greatly on the surface of microplastics
Pathogens such as \emph{Mycobacterium tubercolosis} can aquire mutations which confer antibiotic resistance to rifamycin, the most common treatment for infections with the pathogen. 
Human pathogens has been shown to be enriched on the surface of microplastics, and enrich the presence of antibiotic resistance genes (ARGs) in the microbes.
% Define knowledge gap

Previous studies have commonly looked only at the resistance genes present on microplastics, and not the specific point mutations which may confer resistances.
This work analyzes the metagenome of 395 samples from publically available data to identify point mutations, leading to antibiotic resistance. 
To comprehensively assess point mutation frequencies, the presence of ARGs on microplastics were compared with those found in the surrounding water, and other substrates.

The results suggest that plastics, in general, did not increase the presence of individual resistance-conferring point mutations compared to water,
%percentages!
however specific plastic substrates did carry bacteria with elevated point mutations, such as polyethylene-fiber (PF, +48\%), polyethylene-fiber-polyethylene (PFP, +140\%), polyvinyl chloride (PVC, +25\%), low-density polyethylene (LDPE, +52\%), Ecovio(+103\%), and BI-OPL(120\%). 
% percentages!

Specific substrates (HDPE, PBSA, PFP, PHA, PHBV) were identified by their point mutations of their microbiomes, conferring resistance to fluoroquinolone, rifamycin, aminocoumarin, and rifampicin.
Plastics as a general group had, however, no specific point mutations which would have distinguished it from the water samples, meaning that different plastic substrates induce different point mutations. 
The plastic group as a whole contained more different point mutations than the water group, but there was no increase in specific point mutations. 
This nonspecific increase in the overall level of point mutations could potentially be caused by oxidative stress from additives released from the microplastic particles during weathering and degradation processes, or pollutants adsorbed onto them.
%This work highlights the difference between different plastic substrates, and that care need to be taken to specify the plastic substrate used, since this can significantly impact the results.

This work highlights that several, but not all, plastic substrates increase the presence of specific point mutations and provide insights for further studies aimed at mitigating the impact of antibiotic resistance.

% KEYWORDS (MAXIMUM 10 WORDS)
\vfill
Keywords: antibiotic resistance, plastisphere, point mutations, AMR, ARG, aquatic ecosystems, bioinformatics, mumame.

\newpage				% Create empty back of side
\thispagestyle{empty}
\mbox{}
