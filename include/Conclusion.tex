% CREATED BY MAGNUS GUSTAVER, 2020
\chapter{Conclusion}

%By analysing the number of point mutations which confer antiobiotic resistance, the result of which is shown in figure \ref{hits_type} and \ref{hits_substrate}, one may conclude that the number of point mutations which confer antibiotic resistance are not more prevalent on all plastics, but instead that specific plastic substrates increase this count.
%These substrates include PFP, Ecovio, and BI-OPL. The latter two are biodegradable plastics which contain a blend of PBAT and PLA. 
% \begin{itemize}
    % \item Plastics as a group does not increase the mean mutation percentage of the sample significantly, but specific plastic substrates do.
    % \item When calculating the mean mutation percentage for mutations, grouped by sampletype or substrate, the plastic samples show a significant increase compared to the other two groups. Since the tests are paired, I think this means that more mutations have a higher mean mutation percentage in the plastic samples than the water samples?
    %\item The two above results mean that there are more mutations/hits on the water samples, the total sum is higher from some few which are higher, but that there is a small increase in many different mutations in plastic samples, and therefore they are significant and higher when testing in the second way.
    %\item Random forest shows there are different mutations on different substrates, can distinguish between them.
%\end{itemize}

This study concludes that microplastics do not enrich the overall extent of the prevalence of AMR-conferring point mutations in the microbial communities of the plastisphere in relation to those of the surrounding environments. 
Nevertheless, there are specific plastic types which can enhance such mutations and make the microbes on the plastisphere resistant to fluoroquinolone (HDPE, PFP), rifamycin (PHA), aminocoumarin (PBSA), and rifampicin (PHBV).

In addition, there is an increased number of unique point mutations which confer antibiotic resistance in the microorganisms of the plastisphere, as well as when comparing specific plastic substrates.

In addition, the results provided evidence that some point mutations make their source substrates or medium predictable, including specific substrates (PHBV, PBSA, PHA, and PFP). To predict similar association with a specific group, i.e. water, plastic or non-plastic, is not possible.
Hence, there are point mutations which identify some substrates but not others, and this needs to be taken into account when studing the prevalence of antimicrobial resistance in the plastisphere.

Further studies are needed to investigate why these substrates increase the prevalence of mutations which confer antibiotic resistance, while others do not.
Additional research into the cause of the nonspecific increase of point mutations on plastic substrates is also needed in order to gain knowledge that could help mitigate the threat of antibiotic resistance mutations.
