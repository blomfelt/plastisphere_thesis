\chapter{Discussion}
% According to figures \ref{hits_type} and \ref{hits_substrate}

% By analysing the number of point mutations which confer antiobiotic resistance, the result of which is shown in figure \ref{hits_type} and \ref{hits_substrate}, one may thus conclude that


Based on the result in figures \ref{hits_type} and \ref{hits_substrate} one may thus conclude that the number of point mutations which confer antibiotic resistance are not more prevalent on all plastics, but instead that specific plastic substrates increase this count.
These substrates include PFP, Ecovio, and BI-OPL. The latter two are biodegradable plastics which contain a blend of PBAT and PLA. 

On the other hand, in figures \ref{mean_genes_sampletype} and \ref{mean_genes_substrate}, the result is that the mean mutation percentage is higher for the plastic group when one compares the mean mutation percentage for specific mutations. 

The combination of these 
% seemingly diametrically opposed \todo{disparate? lmao eloquently spoken} 
results means that plastic seems to increases the mutation prevalence for specific genes, but as a general promotor of accumulation of antibiotic resistant mutations.

This claim is supported by the results shown in figure \ref{point_snps}, which show that there are mutations which are more prevalent on plastic substrates. However it may be noted that in this plot, the non-plastic substrates also exhibit a lot of mutations with a high mean mutation percentage, and that there are genes which are prevalent on all substrates.

The results from the random forest analysis show that one cannot easily distinguish the plastic samples from the nonplastic and water samples when grouped by the sample type.
However, there are mutations which are found to be significant when grouped by substrate and may therefore be used to identify which substrate a sample comes from.
One needs to be aware however that a high Mean Decrease of Gini Impurity does not mean that the mutation or AMR gene family has a higher abundance in the corresponding substrate, but only that the variable can be used to differentiate it. The actual abundance may be higher or lower than the reference group, the other substrates, but it does differ from the other substrates. \todo{It can also vary more or less than the reference? Find better reference to refer to}

\section{To-do}
\begin{itemize}
    \item Look at mutations, in what genes are they, what do they do? What the AMR Gene Family does I guess 
\end{itemize}

% This claim is supported by figures \ref{mean_genes_sampletype} and \ref{mean_genes_substrate}, in which the 

% These results are in line with those mentioned above, which is evident when figure \ref{mean_samples_substrate} is taken into consideration.
