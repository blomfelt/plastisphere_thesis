\chapter{Discussion}

% From Máté:
% - Start with a summary of 
%   - what you did
%   - what this project was about
%   - what was the main question
%   - what is you answer to the main question
% - Then you could add a one-one key sentence from each important result section. 
% - Thereafter you "zoom in" and elaborate more for all the sentences that you summarized in 
%   the first paragraph.
%   - And discuss your results, compare them with other findings, state whether the findings were surprising or not, if so, what could explain your observed patterns.
% (also present below)

% - No need to refer to figures in the discussion

% ----- 
% Aim
The aim of the study was to identify specific point mutations in the plastisphere which can lead to altered susceptibility of microorganisms to antibiotics. 
In addition to this the levels of these specific point mutations were to be compared with those in microorganisms which inhabit water or natural substrates.
% Method
To achieve this the point mutations present in samples from 35 different substrates were obtained, in addition to the amount of reads which map to each specific point mutation. 

% Main point
Overall, the results suggest that plastics in general does not increase the prevalence of specific point mutations in microbes in the plastisphere in comparison to those living in planktonic forms in the surrounding water.
However, specific plastic substrates such as PVC, PFP, Ecovio, and BI-OPL increased the relative mutation percentage in the sample.
The antibiotics with the highest mutation percentage were rifampicin, aminocoumarin, vancomycin, and fluoroquinolones.

The presence of certain genes and gene families were successful in indicating sample origin and substrate type.
These include fluoroquinolone resistant parC (HDPE) and rifamycin resistant ropB (PHA) for the gene families and G124S gyrB (aminocoumarin resistance, PBSA), V104I or S84L+D105E parC (fluoroquinolone resistance, PFP), and Q1073R rpoB (rifampicin resistance, PHBV) for the point mutations.
In other words, these genes are more prone to experience point mutations, leading to antibiotic resistance to fluoroquinolone, rifamycin, aminocoumarin, and rifampcin.

% Zoom in and other studies
Previous studies found that the biofilm on microplastics enrich ARGs compared to the surrounding water \cite{zhou2024MicroplasticBiofilmsPromote}, and that this biofilm is distinct from the microbiome of the surrounding water \cite{zadjelovic2023MicrobialHitchhikersHarbouring}. 
% READ PER MILLION
In our study however, there was a higher amount of point mutations in the water samples than the plastic samples, and the freshwater samples had a significantly higher mean than most other substrates. 
The cause of this could be that the main driving force for the accumulation of ARGs in the plastisphere being horizontal gene transfer, and not mutation-induced resistance development, as shown by previous studies \cite{goswami2025MicroplasticsHiddenDrivers} which would explain the relationship we observe.
However, when the amount of point mutations present in different substrates were compared there were plastic substrates (PFP, Ecovio, BI-OPL, PHB, PF, PBAT and LDPE) with more mutations than the water substrates.
This signifies that there are plastic substrates which do increase the prevalence of specific point mutations, which correlates well with the results from Li et al. \cite{li2021ImpactUrbanizationAntibiotic}, which found the amount of ARGs to vary between microplastic substrates. The authors also found that PE had the highest enrichment of aminoglycoside resistance genes, a mutation present in the plastisphere of all the substrates they used.
Some of the plastics we found which increase the amount of point mutations were PFP (PE-fiber-PE), PF (PE-fiber), and LDPE (low-density PE), all of which consists of PE (polyethylene). 
%They also found there to be a significant (p < 0.01) correlation between specific surface area and the abundance of ARGs, which could explain the differences between substrates, however this effect warrants further study. 
% Stevenson et al. \cite{stevenson2024SelectionAntimicrobialResistance} found that different plastic substrates enrich bacteria to different levels, where especially PS particles enrich AMR bacteria. 

% SAMPLES
When instead considering the mean mutation percentage among the samples, the mean mutation percentages was generally lower in the plastisphere than in the surrounding microbial communities.
However, this comparison also showed differences between the substrates, where PVC, PFP, LDPE, Ecovio, and BI-OPL had higher mean mutation percentage than seawater or wastewater. 

% POINT MUTATIONS
When the mean mutation rate of the individual point mutations were compared between the plastic and water group, the plastic group had a higher mean than the water group. 
This is because the plastic samples contained many mutations at low mutation frequencies which the water samples did not, and therefore reduced the mean of the water samples accordingly since these had a mutation rate of zero in the water samples.
This nonspecific increase in mutation frequency could be due to an increase of oxidative stress caused by the pollutants accumulated in the particles or additives released by them \cite{forero-lopez2022PlastisphereMicroplasticsSitu, carvajal-garcia2023OxidativeStressDrives}. In the previous studies, these pollutants included chromium and zinc, and the additives included pigments such as titanium dioxide, and lead(II) chromate \cite{liu2020EffectAgingAdsorption}.

Since it was possible to predict the identity of the substrate to which a sample belonged using specific point mutations or AMR gene families, there was a difference between the different substrates in respect to these. 
However, since one cannot predict association to the plastic group there are no point mutations or AMR gene families which define the plastisphere clearly, highlighting the importance of defining specific substrates when comparing the results of different studies of the plastisphere. 
%The point mutations which may be used to predict identity were present in gyrB, parC and rpoB, all of which 
This result also show there being some substrates which show an increase in specific point mutations, while there is not evidence for the same for the plastic group as a whole.

% The random forest analysis show that there are both AMR Gene Families and point mutations that are significant and may be used to predict the identify of specific substrates, however one cannot predict group association other than to the non-plastic group.

Several of the plastic substrates are biodegradable including BI-OPL, Ecovio, PBAT, PHA, PHB, PHBV, and PLA. Liu et al. \cite{liu2023EffectsComparisonSecondary} show that biodegradable plastics promote the transfer of ARGs between bacteria, however there are no studies that shown an increase in point mutation in them, highlighting another area of further study.

One must take care when interpreting these results, since the mutations identified all come from a single database, which cannot be assumed to be complete. Further studies could use other databases to raise the confidence of the specific mutations found herein. 
Further consideration must also be taken for the clinical relevance of these findings, since the clinical efficacy of the antibiotic resistance is not known, and is only based on presumed conferral of antibiotic resistance, based on the presence of the point mutation in the genome of the pathogen.

The overall result of the current study therefore reject the original hypothesis, there is not an increase in the prevalence of specific point mutations conferring antibiotic resistance in microbes in the plastisphere in comparison to those living in planktonic forms in the surrounding water. However, specific plastic substrates do increase the prevalence of point mutations in the plastisphere, and is an area which warrant further studies. 

% TODO: Most common plastics? -> Theory?

% Obsidian-text: -----------------
%Good Paragraph! 
%Move sentences around and place them in line with your specific findings!
% Previous studies found that the biofilm on microplastics enrich ARGs compared to the surrounding water \cite{zhou2024MicroplasticBiofilmsPromote}, and that this biofilm is distinct from the microbiome of the surrounding water \cite{zadjelovic2023MicrobialHitchhikersHarbouring}. 
    % Stevenson et al. \cite{stevenson2024SelectionAntimicrobialResistance} show that particularly microplastic particles of PS enrich AMR bacteria compared to non-plastic materials such as wood and glass. This falls in line with the result that different plastic substrates behave differently, altough the specific substrate in question was different.% between that study and this one.
%The main driving factor for the accumulation of ARGs in the plastisphere is horizontal gene transfer (HGT) \cite{goswami2025MicroplasticsHiddenDrivers} between microorganisms, and not mutation-induced resistance development.
% This could reduce the selective pressure for point mutations, therefore reducing the amount of them that are present. 
% This could also explain the result of non-selective increase of point mutations, where environmental stresses increase the amount of point mutations non-selectively.
    %Therefore, although the plastisphere selects for AMR-carrying bacteria, the method through which this resistance occur is not, in general, through specific point mutations caused by the plastic substrate.
%The small non-specific increase of point mutations on the plastic substrates could instead be due to oxidative stress caused by the pollutants accumulated in the particles or additives released by them \cite{forero-lopez2022PlastisphereMicroplasticsSitu, carvajal-garcia2023OxidativeStressDrives}.

% From theory:
    %Compared to bacteria in the surroundig water, the rate of gene transfer increased by 7.2-19.6 times \cite{zhou2024MicroplasticBiofilmsPromote} in the plastisphere. Of the tested microplastics in the study, Zhou et al. found that the biofilm formed on polyethylene (PE) plastic had the highest conjugation rate as well as bacterial density. 


% ----- 
% One needs to be aware however that a high mean decrease of Gini impurity does not mean that the mutation or AMR gene family has a higher abundance in the corresponding substrate, but only that the variable can be used to differentiate it. The actual abundance may be higher or lower than the reference group, the other substrates, but it does differ from the other substrates. 
% It can also vary more or less than the reference? Find better reference to refer to

% the antibiotics in 22 resistances:
% - fluoroquinolones
% - rifampicin
% - isoniazid
% - pulvomycin
% - rifabutin
% - enacyloxin iia
% - aminocoumarin
% - fosfomycin
% - vancomycin
% - fusidic acid
% 
% genes: 
% - parC
% - gyrA
% - gyrB
% - rpoB
% - katG
% - EF-Tu
% - parE
% - ponA1
% - soxS
% - UhpT
% - rpoC
% - fusA

% --------------
% Intressant för senare:
% Metagenome Assemeled Genomes - MAG, linking ARGs and other virulence-related genes to their host.
% - Escheria coli harboured more ARGs and virulence factors than any other MAG
% --------------
% HDPE, PBSAA, PHA for AMR Gene Families
% PBSA, PFP, PHBV for point mutations









